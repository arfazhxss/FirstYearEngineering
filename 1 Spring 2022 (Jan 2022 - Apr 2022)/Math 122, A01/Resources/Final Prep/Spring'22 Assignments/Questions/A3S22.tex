\documentclass[11 pt]{article}%
\usepackage{amssymb, amsmath, fullpage}%

\pagestyle{empty}
\lineskip +3pt

\usepackage{tikz}

\pgfdeclarelayer{edgelayer}
\pgfdeclarelayer{nodelayer}
\pgfsetlayers{edgelayer,nodelayer,main}

\tikzstyle{none}=[inner sep=0pt]
\definecolor{hexcolor0xf81e1c}{rgb}{0.973,0.118,0.110}
\definecolor{hexcolor0x3c00ff}{rgb}{0.235,0.000,1.000}

\tikzstyle{whitevertex}=[circle, ,fill=white,draw=black]
\tikzstyle{setcirc}=[circle, ,fill=none,draw=black, scale=11]
\tikzstyle{textbox}=[rectangle, fill=none, draw=none]

\tikzstyle{undirected}=[draw=black]




\begin{document}



\begin{center}
\textbf{{\Large 202201 Math 122 Assignment 3}}
\end{center}

\medskip\noindent\textbf{Due:  Sunday, March 6, 2022 at 23:59}.  Please submit on your section's Crowdmark page.

\bigskip\hrule\medskip\noindent There are five questions of equal value (worth a total of 45 marks), plus a left over question from Assignment 2 (worth 9 marks).   There are 4 bonus marks available if the solutions are typeset with \LaTeX.  Information on obtaining and using \LaTeX is available on the cross-listed Brightspace page.

\medskip\noindent
Please feel free to discuss these problems with each other.  
You may not  access  any ``tutoring'' or ``help'' website in any way. 
In the end, each person must write up their own solution, in their own words, in a way that reflects their own understanding.   Complete solutions are those which are coherently written, and include appropriate justifications.


\medskip\hrule


\begin{enumerate}

\item[$\ast$]  (Assignment 2, Question 5)  Answer each question True or False, and write a sentence or two to briefly explain your reasoning.
Let $A = \{1, \{1\}, 2,   \{\emptyset\}, \{ \{1\}, \{2\}\}, \{\{1\},2\}\}$.  

\begin{enumerate}
\item $\{2\} \in A$
\item $\{1,2\} \subsetneqq A$ 
\item $\{\{1, \{2\}\}\} \subseteq A$
\item $\emptyset \in A$
\item $A \cap \mathcal{P}(A) = \emptyset$
\item $\{2\} \in \mathcal{P}(A)$
\item $\emptyset$ is the only set with no non-empty proper subset.
\end{enumerate}

\item Prove that for all sets $A, B$ and $C$, 
$A \setminus (B \cup C) = (A \setminus B) \cap (A \setminus C)$ by:
\begin{enumerate}
\item using set-builder notation and showing the LHS and RHS are defined by logically equivalent expressions (give reasons for each step);
\item using the Laws of Set Theory  (give reasons for each step);
\end{enumerate}

\item Let $A$ and $B$ be sets.
\begin{enumerate}
\item Prove that if $A \subseteq B$, then $A \setminus B = \emptyset$.
\item Prove that if $A \setminus B = \emptyset$, then $A \cup B = B$.
\item Prove that if $A \cup B = B$, then $A \subseteq B$.
\item Are the three statements (i) $A \subseteq B$, (ii) $A \setminus B = \emptyset$, and (iii) $A \cup B = B$
all logically equivalent?  Explain your reasoning.
\end{enumerate}


\item Let $A$ and $B$ be sets.  Use the same method as in 2. to show that the three statements 
(a) $A = B$,
(b) $A \cup B = A \cap B$, and 
(c) $A \oplus B = \emptyset$
are all logically equivalent.

%Show that the statements (a), (b), (c) below are all logically equivalent.
%\begin{enumerate}
%\item If $A = B$, then $A \cup B = A$,
%\item  If $A \cup B = A \cap B$ then $A \oplus B = \emptyset$.
%\item  If $A \oplus B = \emptyset$, then $A = B$. 
%\end{enumerate}

\newpage
\item   Let $A, B, C$ be sets.  
In each part, if the given statement is true then prove it, and it it is false then give a counterexample. 
In the case that the statement is false and a Venn diagram suggests that one of the sets is a subset of the other, state the relationship that's suggested.
\begin{enumerate}
\item   For all sets $A, B, C$, $(A \setminus B) \setminus C = A \setminus (B \setminus C)$.
\item For all sets $A, B, C$,  $(A \oplus B^c) \oplus C =  A \oplus (B^c \oplus C)$.
\end{enumerate}

\item Let $A_1, A_2, A_3, A_4$ be sets.  
\begin{enumerate}
\item Give a definition of $A_1 \cup A_2 \cup A_3 \cup A_4$ and  $A_1 \cap A_2 \cap A_3 \cap A_4$.
\item Prove that $(A_1 \cup A_2 \cup A_3 \cup A_4)^c = A_1^c \cap A_2^c \cap A_3^c \cap A_4^c$.
\item Write down the dual of the statement in (b).  No proof is required.
\end{enumerate}

\end{enumerate}
\end{document}