\documentclass[12 pt]{article}

\usepackage{amssymb}
\usepackage{amsmath}
\usepackage{amsfonts}
\usepackage{fullpage}
\pagestyle{empty}
\parindent 0pt
\parskip 7pt

\begin{document}
{\bf Math 122 Assignment 3 Solution Ideas}
\begin{enumerate}

\item  [*] (Assignment 2, Question 5)\begin{enumerate}
\item False.  The elements of $A$ are $1, \{1\}, 2,   \{\emptyset\}, \{ \{1\}, \{2\}\}, \{\{1\},2\}$, and $\{2\}$ isn't one of these.
\item True.  Both 1 and 2 are elements of $A$, and $A \neq \{1, 2\}$.
\item False.  As in (a), $\{1, \{2\}\}$ is not an element of $A$,
\item False.  Similar reason as in (a).
\item False.  Both $\{1\} \subseteq A$ and $\{1\} \in A$, so $\{1\} \in A \cap \mathcal{P}(A)$.
\item True.  $\{2\} \subseteqq A$.
\item False. $\{1\}$ has no non-empty proper subset.
\end{enumerate}


\item \begin{enumerate}
\item $ $\\
$\begin{array}{rcl}

A \setminus (B \cup C) &=& \{x: (x \in A) \wedge (x \not\in B \cup C)\}\\
&=& \{x: (x \in A) \wedge \neg (x \in B \cup C)\}\\
&& \quad\quad \mbox{ Definition}\\
&=&  \{x: ((x \in A) \wedge (x \in A)) \wedge ((x \in B^c) \wedge (x \in C^c)\}\\
&& \quad\quad \mbox{ DeMorgan}, \mbox{Conj. Idemp.}\\
&=& \{x: ((x \in A) \wedge (x \in B^c)) \wedge ((x \in A) \wedge (x \in C^c))\}\\
&& \quad\quad\mbox{Associative and Commutative (several times)}\\
&=& \{x: (x \in A \cap B^c) \wedge (x \in A \cap C^c)\} \\
&& \quad\quad\mbox{Definition}\\
&=& \{x: x \in (A \setminus B) \cap (A \setminus C)\} \\
&& \quad\quad \mbox{ Definition}\\
 &=& (A \setminus B) \cap (A \setminus C)
 \end{array}$
 
 \item
$ $\\
$\begin{array}{rcll}
A \setminus (B \cup C) &=& A \cap (B \cup C)^c & \mbox{known}\\
&=& (A \cap A) \cap (B^c \cap C^c) & \mbox{Idempotent}\\
&=& (A \cap B^c) \cap (A \cap C^c) & \mbox{Associative and Commutative (several times)}\\
&=& (A \setminus B) \cap (A \setminus C) & \mbox{known}
\end{array}$
\end{enumerate}

\item \begin{enumerate}
\item 
We prove the contrapositive: if $A \setminus B \neq \emptyset$, then $A \not\subseteq B$.
Suppose $A \setminus B \neq \emptyset$.
Then, by definition, there exists an element of $A$ which is not in $B$.
Thus $A \not\subseteq B$.
Hence if $A \subseteq B$, then $A \setminus B = \emptyset$.
 
 \item  Suppose $A \setminus B = \emptyset$.
 Since $B \subseteq A \cup B$ by definition, it suffices to show that $A \cup B \subseteq B$.
 Take any $x \in A \cup B$.
 Then $x \in A$ or $x \in B$.
 If $x \in B$ there is nothing to show.
 If $x \in A$ then,  since $A \setminus B = \emptyset$, we have $x \in B$.
 In either case $x \in B$.
 Thus $A \cup B \subseteq B$ and hence $A \cup B = B$.
 
 \item Suppose $A \cup B = B$.
 Take any $x \in A$.
 Then $x \in A \cup B$ by definition of union.
 Since $A \cup B = B$, we have $x \in B$.
 Therefore $A \subseteq B$.
 
 \item Yes.  In parts (a), (b), (c) above we have shown (i) $\Rightarrow$ (ii),  (ii) $\Rightarrow$ (iii), and (iii) $\Rightarrow$ (i).
 The three statements are then logically equivalent by Assignment 1, Question 3.
 \end{enumerate}
 
 \item\begin{enumerate}
 \item We need to show (a) $\Rightarrow$ (b), (b) $\Rightarrow$ (c), and (c) $\Rightarrow$ (a).
 
 \smallskip
(a) $\Rightarrow$ (b). Suppose $A = B$.  Then $A \cup B = A \cup A = A = A \cap A = A \cap B$.

\smallskip
(b) $\Rightarrow$ (c). We prove the contrapositive: If $A \oplus B \neq \emptyset$ then $A \cup B \neq A \cap B$.
Suppose $A \oplus B \neq \emptyset$.
Then there exists $x \in A$ such that $x \not\in B$, or there exists $x \in B$ such that $x \not\in A$.
In either case, the element $x$ belongs to $A \cup B$ but not to $A \cap B$.
Thus  $A \cup B \neq A \cap B$.
Hence if $A \cup B = A \cap B$, then $A \oplus B = \emptyset$.

\smallskip
(c) $\Rightarrow$ (a).
We prove the contrapositive: if $A \neq B$ then $A \oplus B \neq \emptyset$.
Suppose $A \neq B$.
Then either there exists $x \in A$ such that $x \not\in B$, or there exists $x \in B$ such that $x \not\in A$.
In either case, $x \in A \oplus B$, so $A \oplus B \neq \emptyset$.
Hence if $A \oplus B = \emptyset$, then $A = B$.

\smallskip
The proof is now complete.
 \end{enumerate}

\item \begin{enumerate}
\item It is possible to generate a counterexample from a Venn diagram, as shown in class.  Here is a 
different one.  Let $A = B = C = \{1\}$.
Then $A \setminus (B \setminus C) = A \setminus \emptyset = A = \{1\}$, while
$(A \setminus B) \setminus C = \emptyset \setminus C = \emptyset \neq \{1\}$.
Thus $A \setminus (B \setminus C)$ is not equal to $(A \setminus B) \setminus C$ for all sets $A, B, C$.
But the Venn diagram suggests $(A \setminus B) \setminus C$ is a subset of $A \setminus (B \setminus C)$.

\item The two sets are equal. We show 
LHS $\subseteq$  RHS and RHS $\subseteq$ LHS.

\smallskip
LHS $\subseteq$  RHS.
Take any $x \in (A \oplus B^c) \oplus C$.
Then $x \in A \oplus B^c$ and $x \not\in C$ or $x \not\in A \oplus B^c$ and $x \in C$.
In the first case $x \not\in C$ and either $x \in A$ and $x \not\in B^c$, or $x \in B^c$ and $x \not\in A$.
In all cases, $x \in A \oplus (B^c \oplus C)$.

\smallskip
RHS $\subseteq$ LHS.
Take any $x \in  A \oplus (B^c \oplus C)$.
Then $x \in A$ and $x \not\in (B^c \oplus C)$, or  $x \not\in A$ and $x \in (B^c \oplus C)$.
In the first case, $x \in A, x \in B^c$ and $x \in C$, or $x \in A, x \not\in B^c$ and $x \not\in C$.
In the second case $x \not\in A, x \in B^c$ and $x \not\in C$, or $x \in A, x \not\in B^c$ and $x \not\in C$.
In all cases, $x \in (A \oplus B^c) \oplus C$.

Thus $(A \oplus B^c) \oplus C =  A \oplus (B^c \oplus C)$.

 \end{enumerate}
 
 
 \item \begin{enumerate}
 \item Let $A_1, A_2, A_3$ and $A_4$ be sets.  We define
  $$A_1 \cup A_2 \cup A_3 \cup A_4 = \{x: (x \in A_1) \vee (x \in A_2) \vee (x \in A_3) \vee (x \in A_4\}$$
  and 
 $$A_1 \cap A_2 \cap A_3 \cap A_4 = \{x: (x \in A_1) \wedge (x \in A_2) \wedge (x \in A_3) \vee (x \in A_4\}.$$
 
 \item We show LHS $\subseteq$  RHS and RHS $\subseteq$ LHS.
 
 \smallskip
LHS $\subseteq$  RHS.
Take any $x \in  (A_1 \cup A_2 \cup A_3 \cup A_4)^c$.
Then, by definition, $x \not\in A_1$, $x \not\in A_2$, $x \not\in A_3$ and $x \not\in A_4$.
That is,  $x \in A_1^c$, $x \in A_2^c$, $x \in A_3^c$ and $x \in A_4^c$.
Therefore, $x \in (A_1^c \cap A_2^c \cap A_3^c \cap A_4^c)$.

\smallskip
RHS $\subseteq$ LHS.
Take any $x \in  (A_1^c \cap A_2^c \cap A_3^c \cap A_4^c)$.
Then, by definition, $x \not\in A_1$, $x \not\in A_2$, $x \not\in A_3$ and $x \not\in A_4$.
Therefore, $x \not\in (A_1 \cup A_2 \cup A_3 \cup A_4)$.
Hence $x \in (A_1 \cup A_2 \cup A_3 \cup A_4)^c$.

\smallskip
Therefore $(A_1 \cup A_2 \cup A_3 \cup A_4)^c = (A_1^c \cap A_2^c \cap A_3^c \cap A_4^c)$.

\item $(A_1 \cap A_2 \cap A_3 \cap A_4)^c = (A_1^c \cup A_2^c \cup A_3^c \cup A_4^c)$.

 \end{enumerate}
 
 
\end{enumerate}
\end{document}
